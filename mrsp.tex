\documentclass[10pt,letterpaper]{article}
\usepackage[top=0.85in,left=2.75in,footskip=0.75in]{geometry}

% amsmath and amssymb packages, useful for mathematical formulas and symbols
\usepackage{amsmath,amssymb}

% Use adjustwidth environment to exceed column width (see example table in text)
\usepackage{changepage}

% cite package, to clean up citations in the main text. Do not remove.
\usepackage{cite}

% Use nameref to cite supporting information files (see Supporting Information section for more info)
\usepackage{nameref,hyperref}

% line numbers
\usepackage[right]{lineno}

% ligatures disabled
\usepackage{microtype}
\DisableLigatures[f]{encoding = *, family = * }

% color can be used to apply background shading to table cells only
\usepackage[table]{xcolor}

% array package and thick rules for tables
\usepackage{array}

% Use listings instead of verbatim
\usepackage{listings}
\lstset{
  basicstyle=\ttfamily,
  escapeinside=!!
}

% enumerate package lets us use letters instead of numbers
\usepackage{enumerate}

% create "+" rule type for thick vertical lines
\newcolumntype{+}{!{\vrule width 2pt}}

% create \thickcline for thick horizontal lines of variable length
\newlength\savedwidth
\newcommand\thickcline[1]{%
  \noalign{\global\savedwidth\arrayrulewidth\global\arrayrulewidth 2pt}%
  \cline{#1}%
  \noalign{\vskip\arrayrulewidth}%
  \noalign{\global\arrayrulewidth\savedwidth}%
}

% \thickhline command for thick horizontal lines that span the table
\newcommand\thickhline{\noalign{\global\savedwidth\arrayrulewidth\global\arrayrulewidth 2pt}%
\hline
\noalign{\global\arrayrulewidth\savedwidth}}

\usepackage{color}

% Remove comment for double spacing
%\usepackage{setspace}
%\doublespacing

% Text layout
\raggedright
\setlength{\parindent}{0.5cm}
\textwidth 5.25in
\textheight 8.75in

% Bold the 'Figure #' in the caption and separate it from the title/caption with a period
% Captions will be left justified
\usepackage[aboveskip=1pt,labelfont=bf,labelsep=period,justification=centering,singlelinecheck=off]{caption}
\renewcommand{\figurename}{Fig}

% Use the PLoS provided BiBTeX style
\bibliographystyle{plos2015}

% Remove brackets from numbering in List of References
\makeatletter
\renewcommand{\@biblabel}[1]{\quad#1.}
\makeatother

% Leave date blank
\date{}

% Header and Footer with logo
\usepackage{lastpage,fancyhdr,graphicx}
\usepackage{epstopdf}
\pagestyle{myheadings}
\pagestyle{fancy}
\fancyhf{}
\setlength{\headheight}{27.023pt}
\rfoot{\thepage/\pageref{LastPage}}
\renewcommand{\footrule}{\hrule height 2pt \vspace{2mm}}
\fancyheadoffset[L]{2.25in}
\fancyfootoffset[L]{2.25in}
\lfoot{\sf PLOS}

% Allow Markdown while prototyping
\usepackage[hashEnumerators,smartEllipses]{markdown}


\begin{document}
\vspace*{0.2in}

\begin{flushleft}
{\Large
\textbf\newline{Good Enough Practices for Managing Research Software Projects}
}
\newline
\\
{Greg Wilson}\textsuperscript{1{\ddag}}
\\
\bigskip
\textbf{1} Third Bit\\
{\ddag} Corresponding author, gvwilson@third-bit.com.
\end{flushleft}

\section*{Introduction}

Many widely-used pieces of research software begin as someone's personal project.
As the number of users grows,
the original creator often finds themself spending more time on email than writing code.
As it becomes even more popular
they find themself trying to balance stability against progress
or 

\cite{Wilson2014,Wilson2017,Irving2021} focused on the technical aspects of building research software,
such as version control and automating repeated tasks,
but also touched on some management aspects:
creating a shared to-do list,
deciding on communication channels,
and so on.

This article introduces practices for managing a team of a dozen people working together to build research software.
  - Inspired by \cite{Barker2010,ODuinn2021}
- You don't have to invent this yourself \url{https://www.askamanager.org/}

\subsection*{Acknowledgments}

The author is grateful to Daniel Standage for helping create
the first version of the workshop from which this paper is derived.

\section{When to Start}

- Brooks advocated a ``chief surgeon'' model in the 1970s \cite{Brooks1995}
  - Often referred to in open source as ``benevolent dictator''
  - But as a Latin American colleague pointed out, there's no such thing{\ldots}
- Disparaged since then
  - But 80\% of projects on GitHub are ``hero projects'' \cite{Majumder2019}
  - 5\% or less of people responsible for 95\% or more of interactions
  - ``Heroes'' commit far fewer bugs than other contributors
- Common and sensible in young or small projects
  - Especially where specialized domain knowledge is required
  - Or where one person (the PI) actually does have all the authority
  - But brittle (founder can move on)
  - Leads to emergence of unofficial (i.e., unaccountable) sub-leaders as project grows
  - As project grows to include other contributors, they'll want a say
  - This paper is about what happens when you've outgrown that model

\subsection*{What \emph{Is} a ``Project''?}

- A dataset used by several groups in several ways
  - Has its own data collection and tidying scripts
- A one-of-a-kind analysis
  - Data subsets, Jupyter notebooks, generated PDFs
- A software package
  - With a tutorial, documentation, and sample data

- One repository per publication
  - Only if datasets and tools are their own projects
- One per tool
  - Only if you are comfortable creating packages
- One per team
  - But teams change over time
- One per regular meeting

\section{How to Think About Your New Role}

1. Recalibrate your productivity sensors
   - Your productivity is no longer how much code you wrote, how many samples you analyzed, or how many papers you wrote
   - It is how many obstacles and distractions your team *didn't* have to deal with so that they could do these things
2. Your roles (plural) are all about managing risk

| Department         | What It Does                            | What Risk It Mitigates                                |
| ------------------ | -------------------------------------   | ------------------------------------------------------|
| Project Management | Keep the team on track                  | We aren't doing the right things together             |
| Product Management | Translate needs into features           | We're building the wrong thing                        |
| Human Resources    | Getting and keeping contributors        | We don't have the right people / someone left         |
| Marketing          | Make people aware of how you can help   | People don't know we exist or how we can help them    |
| Sales              | Help people adopt your solution         | People know you're there but don't use what you build |
| Support            | Help people who are using your solution | It's too hard to start/keep using what we build       |
| Finance            | Manage cashflow                         | We don't know what money we have                      |

- Notes
  - Finance isn't about getting money: that's Sales
  - Won't discuss finance further in this paper

\section{Governance}

- Every group has a power structure
  - Only question is whether it is explicit and accountable, or implicit and unaccountable \cite{Freeman1972}
- https://communityrule.info/ describes some options and \cite{Fogel2021} describes some more
- GitHub's Minimum Viable Governance (https://github.com/github/MVG) guidelines ducks the problem

\begin{quotation}
  \textbf{2.1. Consensus-Based Decision Making}.
  Projects make decisions through consensus of the Maintainers.
  While explicit agreement of all Maintainers is preferred, it is not required for consensus.
  Rather, the Maintainers will determine consensus based on their good faith consideration of a number of factors,
  including the dominant view of the Contributors and nature of support and objections.
  The Maintainers will document evidence of consensus in accordance with these requirements.
\end{quotation}

- In practice, ``consensus-based'' usually means ``governance by the self-confident and stubborn''
  - Excludes a lot of people
- Elected representation at the very large end
  - With explicit rules for suffrage
- In between, best to use Martha's Rules \cite{Minahan1986}
- Anyone can put forward a proposal by filing it at least 24 hours before a scheduled meeting
  - One-line summary
  - Background
  - Concrete proposal
  - Pros and cons
  - Alternatives
- At most two pages
  - Preferably shorter
- A quorum is established if half or more of voting members are present (see below)
- The meeting may not discuss or vote on a proposal unless its sponsor (or their delegate) is present.
- The sponsor presents the proposal
  - Usually unnecessary since everyone should have read it
- Members cast a *sense vote*: support, neutral, or oppose
  - If everyone supports or is neutral, go immediately to a binary vote with no further discussion
  - If a majority is opposed or neutral, send proposal back to sponsor for further work
- If a minority of members oppose, set a timer for 10 minutes of moderated discussion
- Then call a final binary vote in which everyone must support or oppose
  - No neutral votes allowed, since either we're going to do this or not
- If a majority support, the proposal is accepted
  - Otherwise, it is returned to the sponsor for further work

\begin{quotation}
- Needs to be an explicit rule about how to become a member
  - And a rule about when and how someone stops being a member (FIXME: forward ref to HR)
- This may align with who gets authorship credit on software publications
\end{quotation}

FIXME: example

\section{Project Management}

- Project manager is responsible for the schedule
  - Who's doing what, in what order
  - What do we cut or scale back when (not if) we fall behind
  - ``A schedule's purpose is to tell us how far behind we are and what to do about it''
- Planning
  - Sticky notes on a whiteboard identifying the proposals, issues, etc.
  - Lots of discussion (mostly about priority and dependencies)
- Build a schedule that guarantees some successes
  - Tackle big things early in the cycle and fall back to smaller ones if need be
- Once you're rolling: progress, plans, and problems go up; context, responsibilities, and priorities come down
- Reporting up:
  - ``This is what I did''
  - ``This is what I'm going to do between now and the next check-in''
  - ``This is what prevents me (or could prevent me) from making progress''
  - FIXME: example
- Reporting down:
  - ``This is the bigger picture (so that you can make decisions that align with overall goals)''
  - ``This is what you're supposed to deliver''
  - ``This is what's most important'' (since there's always more work than time)
  - FIXME: example
- Make these findable \cite{Lin2020}
  - Team can browse one another's status and problems (solves a lot of the latter)
  - They can pick up if someone else leaves

\subsection*{Use issues as a shared to-do list}

- ``Version control tells you where you've been; issues tell you where you're going.''
- An issue's one-line title is its most important feature
- Its tags are almost as important
  - Kind: \texttt{bug}, \texttt{feature}, \texttt{discussion}, \texttt{action}
  - State: \texttt{current}, \texttt{in-progress}, \texttt{ready-to-review}
  - Reason: \texttt{finished}, \texttt{wont-fix}, \texttt{duplicate}
- Don't use for releases
  - Quickly becomes overwhelming
- Project lead should triage bugs weekly
  - What should be closed?
  - What should be assigned?
- Maintain the list of tags and their meanings
- Recalibrate your productivity metrics
  - This is now your job
- Assign PRs to people for review
  - Remind people to claim issues and mark as \texttt{in-progress} when they create branches
- Do reviews yourself
- Handle merges to \texttt{main}
- Prune old branches and stale PRs

\subsection*{Roles and Responsibilities}

- Team members should interact based on roles
  - Because otherwise it becomes ``knowing the right person'' which leaves some people out in the cold
  - Also helps ensure continuity
    - ``Nobody can do X because it's under the personal account of someone who has left the team''
- Define a small number of roles
  - Google Docs gets by with viewer, reviewer, and author
  - Your project shouldn't need more than 4

| Role        | Capabilities       |
| ----------- | ------------------ |
| contributor | create issues      |
|             | comment on issues  |
|             | create PRs         |
|             | review PRs         |
|             | create proposals   |
|             | vote on proposals  |
| admin       | merge PRs          |
|             | close issues       |
|             | create releases    |
|             | publish blog posts |
|             | decide authorship  |
| arbitrator  | Code of Conduct    |

- Then assign one or more roles to individuals

| Person    | Role        | Note     |
| --------- | ----------- | -------- |
| mhamilton | contributor |          |
| bwk       | contributor |          |
| kjohnson  | contributor |          |
|           | admin       |          |
| ghopper   | arbitrator  | external |

- Bonus: doing this gives you a list of what actually needs to be done
  - It's always longer than you first expect
- Helps with succession planning
  - "We don't have anyone who does that any more…"
- And holidays

\subsection*{Checklists}

\cite{Gawande2007} popularized the idea that using checklists improves
results even for experts, and \cite{Aveling2013,Ramsay2019} confirmed
their effectiveness. While \cite{Hatton2008} found no evidence that
they made a difference to code reviews by professionals, they help a
lot with project management. Every role should have a set of short
checklists for routine tasks.

\section{Product Management}

- The product manager manages the feature list
  - What can't we do?
  - What are we doing that hurts?
- Requires deep domain knowledge
  - And a willingness to ask people who don't push their opinions forward
  - Because ``dark matter developers'' aren't typical \cite{Hanselman2012}
- Note that ``improving'' doesn't always mean ``adding features'' \cite{Perri2018}
  - Most features are never used by most people \cite{Xu2015}

\begin{verbatim}
NAME
     ls -- list directory contents

SYNOPSIS
     ls [-ABCFGHLOPRSTUW@abcdefghiklmnopqrstuwx1\%] [file ...]
\end{verbatim}

Remember that not writing software takes less time.  \cite{Sedano2017}
found that software development projects have nine types of waste:
building the wrong feature or product, mismanaging the backlog,
rework, unnecessarily complex solutions, extraneous cognitive load,
psychological distress, waiting and multitasking, knowledge loss, and
ineffective communication.  *None of these are software issues,* so if
you only think about the software side of your project, it's going to
take longer and hurt more than it needs to.

\section{Human Resources}

- Discussion elsewhere about recruiting volunteers, making them productive, managing open communities \cite{Sholler2019}
  - So focus on hiring staff once you have budget
- People do lots of simple things wrong \cite{Behroozi2020}

1. It's a legal matter
   - Every jurisdiction has rules about hiring people
   - Institutions often have union agreements or contractual obligations
   - So ask Human Resources for mentoring
     - Even if you've gone through it before, because rules can change
   - They'll be grateful you came to them before things went wrong
2. - Always post positions
   - One person's ``I know a guy'' is another's ``I wasn't even invited''
3. Make the details clear
   - Do they have or need work authorization?
   - Are they willing to relocate?
   - Do they actually have the right skills for this job?
   - Include timelines in the job ad
   - Describe the process and the kinds of questions they'll be asked
     - [Automattic](https://automattic.com/work-with-us/how-we-hire-developers/) is a good model
   - FIXME: competency matrix
4. Interview scheduling
   - Use [Calendly](https://calendly.com/) or a similar service
   - If you have to reschedule, give candidates several days' notice
     - If you have to reschedule twice, you're telling them you don't want them
5. Interviews
   - Stick to the advertised process
     - Role-play a practice interview if you can
   - If you don't know enough to judge the candidate, find someone who can
   - Make sure they're paying attention (rather than answering email)
6. Hearing back
   - Status updates
     - Give candidates a timeline and regular updates
     - Don't try to hide the fact that you're interviewing other people
   - Give feedback to unsuccessful candidates
     - ``You're on the right track, but need to have done more with AWS and Kubernetes''
     - People aren't always grateful for it, so it's best to do by email
7. Offer and negotiation
   - Let HR handle it

- Firing is more complicated
  - Sometimes people have good reasons for poor performance
  - Sometimes they don't
  - This is the hardest part of leading a project
  - But keeping someone on the team who shouldn't be there is bad for morale as well as productivity
1. Legal requirements
   - Every jurisdiction has rules about firing people, too
   - Again, talk to someone in Human Resources before doing anything else
2. Check with someone
   - Hard to work with someone you dislike
     - But not liking someone isn't sufficient reason to fire them
   - Talk to someone outside the project
     - More likely to be objective
     - Less likely to let something slip
     - Less likely to feel pressured to take your side because you're the boss
3. Create a transition plan
   - It's awkward to fire the only person who knows the password to the production server
     - And risky as well
   - Suddenly asking someone to document their work is a give-away
   - Better to ask everyone to do it all the time
     - Which helps keep roles and responsibilities up to date
   - Write out the steps you'll take immediately after breaking the news
     - Change passwords on servers
     - Remove from GitHub group
     - Remove from mailing list
     - Return loaned hardware
   - All of this needs to be done when someone leaves voluntarily
     - And helps you realize just how much ``stuff'' your project has
4. Make sure it's not a surprise
   - If someone is surprised they're being fired, *you* have made a mistake
   - If it's for poor behavior:
     - They should know what's in the [Code of Conduct](../conduct/)
     - And you should have given them warnings for minor violations
   - If it's for poor performance:
     - You should have told them that they weren't meeting expectations{\ldots}
     - {\ldots}given them a chance to explain{\ldots}
     - {\ldots}and given them an opportunity to improve
   - You will feel pretty bad if they've been missing deadlines because of a family illness
5. Delivery
   - Write out what you're going to say
   - Keep it short
   - Practice it a few times
   - Get straight to the point
   - Don't get drawn into discussion of ``what if''
   - Stick to the practical matters you identified in the transition plan
6. Tell the team right away
   - Before rumors start to circulate
   - Keep statement brief and to the point
   - Do *not* discuss details
     - The person you fired has a right to privacy
     - And you don't want other team members worrying that you'll say something about them some day
7. Keep a record of all communication
   - It's hard for people to think clearly when you're angry or hurt
     - And they may not act in good faith
   - Communicate by email
   - If they insist on a call:
     - Ask to record it and give them permission to do the same
     - Or have a third party present
8. Remember that it's OK to cry
   - Because we all do

\section{Marketing}

- Put open access versions on the web
  - Please use preprint servers like arXiv or bioRxiv rather than (just) your own website
  - Because paywalls are the raised middle finger of academia and you might move
- Give every report a DOI (Zenodo) does this for free
- Give datasets and software releases DOIs as well
  - GitHub-Zenodo integration makes this easy
- Make everyone on your team citable
  - People change names, institutions, email addresses, genders, titles{\ldots}
  - Get an ORCID and include it in all publications

- Your actual product is your grant proposals \cite{Kuchner2011}
  - That's what people give you money for
- Your papers are ``just'' proof that you're worth funding
  - Because funders have a signal-to-noise problem too

> Making sure that the people who ought to know
> how you can make their lives better
> actually do know.

- If what you offer isn't actually useful, persuading people relies on psychological tricks
  - Manufacturing desire \cite{Berger2008,OReilly2010}
- If what you offer *is* useful, the challenges are:
  1. Thinking like someone else
  2. Being heard above the noise
- For example (FIXME: modify to writing/reading software)
  - Person writing a paper:
    - Are the error bars right?
    - Do my co-authors agree with the conclusions?
    - Will this be accepted before someone else gets the same result?
  - Person reading a paper:
    - Did the authors cite my work?
    - Does this paper support my current research direction?
    - Can I use their data or software to speed up my work?
- Create a persona to capture details of who you're trying to reach

> Dalha, 58, is a professor in geophysics and the Dean of Science at a large state university.
> She is evaluated annually on (a) how many high-citation publications faculty produce,
> (b) whether research funding is increasing year-on-year, and
> (c) how many of the university's PhD graduates are offered tenure-stream positions at top-10 schools.
> Dalha is passionate about getting more young women into STEM,
> and worried by the number of faculty and grad students
> who are leaving the university for industry.

- Craft your message

1. For ``description of target audience''
1. who want ``statement of their needs'',
1. ``project name'' provides ``statement of key benefits''.
1. Unlike ``specific alternative(s)'',
1. we ``key differentiator''.

FIXME: example for research software

- Getting the message to peers
  - Create a project-specific blog and Twitter account
    - But don't expect much traffic
    - Real purpose is the appearance of depth
  - Contributing to established channels has more impact
    - Guest post on one of the blogs you read
    - Talk at established conference
    - RSE discussion session,
      guest lectures in someone else's course,
      etc.
- Getting the message to funders
  - Don't start when your proposal goes in
  - Be part of something larger
    - Borrow legitimacy
    - Note: if you're a senior figure, please lend legitimacy when you can
- Contribute to discussion papers, requests for comment, etc.
- Collect data but tell stories

- A brand is the first thing people think of when they think of something
  - A recognizable logo
  - A short, simple tag line

> 1. Be first.
> 2. If you can't be first, create a new category to be first in.

\section{Sales}

- Don't have ``sales'' for open research software
  - Turning your research software into a business is another paper entirely
- But still need to persuade interested parties to give it a try

1.  Don't.
    - If you have to talk someone into something, odds are that they don't really want to do it
    - Respect that
      - Better to leave something undone than to engender resentment
2.  Be kind
    - Lots of books on sales (and dating) teach psyhological manipulation
      - Only works once
      - You don't want to be that person
    - Kindness is usually reciprocated
      - And even if it's not, at least you were kind
3.  Appeal to the greater good
    - Don't open with ``here's what's in it for you''
      - Signals that they should be seeking personal advantage (possibly at your expense)
    - Instead, explain how you and they are going to make the world a better place
    - And mean it
      - If what you're proposing *isn't* going to make the world better, propose something else
4.  Start small
    - Most people are reluctant to dive into things head-first
    - So give them a chance to test the waters and to get to know you
    - It will often end there
      - Everyone is busy, tired, or has projects of their own
5.  Build a community
    - The Carpentries isn't just about teaching, and SciPy isn't just about numerical computing
    - They're places where people can hang out with other people who share their hopes and values
    - Find one if you can, build one if you have to
6.  Establish a point of connection
    - ``I was speaking to X'' or ``we met at Y'' gives people context
      - Makes them more comfortable
      - Makes you more credible
    - Be specific
      - ``I recently came across your website'' isn't convincing
7.  Establish your credibility
    - Who are your backers?
    - How big are you?
    - How long have you been around?
    - Joining forces with an existing movement helps
    - Borrow legitimacy (again, if you're established, lend if you can)
8.  Tell people what you want as well as what you offer
    - Because who knows?
    - It's also a sign of respect
      - Telling people you're moving a few boxes when you're packing up an entire house only works once
9.  Create a slight sense of urgency
    - ``We're going to launch this in the spring'' vs. ``We'd eventually like to launch this''
    - The word ``slight'' is important
      - If your request is urgent, people will assume you're disorganized or that something has gone wrong
      - And will then err on the side of prudence
10. Take a hint
    - If the first person you ask says ``no'', ask someone else
    - If the fifth or the tenth person says ``no'', ask yourself:
      - Am I asking the right way?
      - Am I asking for the right things?
      - Is this a reasonable goal?
      - Is this a worthwhile goal?

> *The 90-9-1 rule*
>
> - 90\% of people will watch
> - 9\% will speak up
> - 1\% will actually do things
> - Set your expectations accordingly

\section{Support}

- ``Marketing before, documentation after''
  - What you write to get attention is not what you need to help people who are now using your software
- You can't afford support staff, so you'll have to rely on:
  - Documentation
  - Discussion forums
  - Discussion channels
- Define your audience
- A novice doesn't yet have a mental model of the domain
  - Doesn't know what they don't know
  - Needs a tutorial that introduces and illustrates key ideas
    - Leave out the details and special cases for now
    - Use authentic tasks
- A competent practitioner can do routine tasks with routine effort
  - This is as far as most people need to go in most fields
  - Will find novice tutorials frustrating
    - ``I already know that''
  - Needs reference guides, cookbooks, and Q\&A sites
    - Can recognize what they need when they see it
- An expert can solve routine problems at a glance and get the right answer for one-in-a-thousand cases
  - Pattern matching rather than reasoning
  - Distinguish between domain experts and experts with your software
  - Need the same material as competent practitioners
    - Nobody's memory is perfect
  - May also be interested in essays explaining why the code is built the way it is

> **False beginners**
>
> - Someone who appears to know every little but can transfer understanding from another domain
>   - E.g., has never used Python but has done a lot of R or vice versa
>   - Knows how to program, just not in this language
> - Will find tutorials for true beginners frustratingly slow{\ldots}
> - {\ldots}but will struggle with reference guides because they don't (yet) have the new vocabulary
> - Ideal solution is ``X for Y'' guide
>   - But we usually don't have enough time

- README
  1. Explain what the software does
  2. List required dependencies
     - In a format that can be consumed by a package manager
  3. Provide compilation or installation instructions
  4. List all input and output files
  5. List a few example commands to get a user started quickly
  6. State attributions and licensing
- Embedded documentation
  - Put specially-formatted comments or strings in source files
    - The closer docs are to code, the less likely they are to fall out of step
  - Have the build tool extract and format them
    - Automatically create cross-references and an index
- When doing exploratory programming, write a brief note to remind yourself of each file or function's purpose
  - Can tell these are useful if we can read them aloud in the order the functions are called in place of the function's name and parameters
  - Begin with an active verb
    - `Extracts`, `normalizes`, `plots`, etc.
    - E.g., `Find current age in years from birth date.`
    - Or, `Clip signals to lie in [0...1].`
  - Describe how inputs are turned into outputs and/or what side effects the function has
    - If the function does both, consider rewriting it
- Release
  - The name and purpose of every public class, function, and constant
  - The name, purpose, and default value (if any) of every parameter
  - Any side effects the function has
  - The type of value returned by every function
  - What exceptions those functions can raise and when
- However{\ldots}
  - Many technical writers have a dim view of embedded documentation
    - ``Never actually up to date''
    - ``All the bricks but none of the buildings''
  - Prefer separate files with long-form explanations
    - Use a static site generator
  - Re-run examples embedded in docs automatically to check consistency
    - Computational notebooks
- Creating a useful FAQ
  - Hard for creators to guess what questions newcomers will actually have (expert blind spot)
  - Harvest questions from discussion forums and online searches
    - Allow people to tag issues as ``question''
    - Turn each one into an FAQ entry or respond with a link to an existing entry
  - Establish a tag on Stack Overflow
    - Check regularly and make sure answers are the ones you want
- Asking a good question
  - Write the most specific title you can
    - ``Why does division sometimes give a different result in Python 2.7 and Python 3.5?'' is much better than, ``Help! Math in Python!!''
  - Give context before giving sample code
    - A few sentences to explain what we are trying to do and why helps people determine if their question is a close match or not
  - Provide a minimal reproducible example (reprex)
    - Greatly increases the chances of a useful answer
  - Keep each item short
    - May feel like trivializing
    - But is easier to keep up to date
  - Provide multiple explanations at different levels
    - Brevity of tldr.sh followed by lengthier explanation for newcomers
- FIXME: who creates this and when?

\section*{Conclusion}

Here's what done looks like:

1.  A shared Google Drive with a doc called ``Roles and Responsibilities''
    - Defines roles and explains what each is responsible for in one page
    - Each role has a doc with its checklists
    - Google Doc because some collaborators aren't comfortable with Git
      - And to make it easier to paste in figures and screenshots
1.  Same shared Google Drive has one doc per year called (e.g.) ``Progress 2022''
    - Section headings are weekly meeting dates
      - Table for each week with columns Name, Progress, Plans, and Problems (bullet points)
      - Anything too long to fit comfortably in the table is linked to an issue in the project's GitHub repository
    - Project has a little script that lists issues and PRs touched by each person (reminder)
1.  Weekly hour-long status meeting (which often finishes early)
    - On Wednesday so that people aren't scrambling on Friday or a weekend (or holiday) to write status updates
    - Rotating moderator: last week's moderator is this week's note-taker
    - Before meeting, members star points in the status doc they want to discuss
    - Moderator draws up agenda based on starred items
1.  Proposals can be done as either Google Docs (in shared folder) or GitHub issues
    - Must be flagged to moderator the day before the meeting for inclusion
    - Added to agenda
1.  Project has a single repo with code, website, tutorials, etc.
    - So that releases are in sync
1.  Uses Google Docs (again) for publicity materials (because non-programmers)
    - All materials are owned by project account, not personal accounts
    - Every change larger than a typo produces a new doc
    - Every doc has date in title, e.g., ``University Press Release 2022-05-13''
1.  Budgets for grant proposals, job contracts, etc., are stored in university system
    - Legal requirement
1.  \texttt{GOVERNANCE.md} in root directory of project explains Martha's Rules
    - And membership rule: anyone who has had a PR merged in the last year or made some other significant contribution (as determined by the PI)
    - List of active members and alumni is in the foot of \texttt{GOVERNANCE.md}
1.  Another small script checks that the tags in each project repository are consistent and that each issue has at least one tag
1.  Project website has a ``skills ladder'' on the ``Positions'' page (even when positions aren't open)
    - ``What we mean by each of these terms for the research and coding tracks''
1.  Project website has a value statement and a contact address that \emph{isn't} anyone's personal address
    - Plus a page for publications
    - Plus a page pointing at all repositories
    - Plus a ``Getting Started'' page
    - And a ``Who's Using Us How'' page
    - And a ``People'' page
1.  The ``help'' option for the software includes the URL to the project page

\section*{Appendix: How to Run Progress Meetings}

1. Does there actually need to be a meeting?
   - To *inform*? Only if you are expecting questions
   - To *consult*? Only if people get a vote
     - Otherwise it's just informing with pretense
   - To *discuss* and *make decisions*? Yes
     - But only in small groups
     - Or with well-defined procedural rules
   - To *brainstorm* or *collaborate*?
     - That's a very different kind of meeting
2. Create an agenda
   - If you don't care enough to make a list, you don't need a meeting
   - Start with status reports
   - Include timings
   - Prioritize
   - Plan to end early
     - ``The most fundamental unit of time is the bladder''
3. Choose a moderator
   - The moderator should *not* do most of the talking
     - Any more than the conductor plays most of the notes
   - Call on specific people in order
   - Allow them one point at a time
   - Moderator should keep a backlog
4. Require politeness
   - All the other rules are special cases of this{\ldots}
   - No technology during in-person meeting
     - Except for assistive technology or family need
     - ``Please put your devices in politeness mode''
   - No interruptions
     - Except by moderator
   - No rambling
   - You *do* have a Code of Conduct, right?
5. Record minutes
   - So people who weren't there know what happened
   - So people who were there agree what happened
   - So people can be held accountable at later meetings
6. Be an active participant
   - Decline invitations
     - *If* you agree to abide by what the meeting decides
   - Read the agenda and material before the meeting
   - Take your own notes
     - Just like you would in class, and for the same reason \cite{Aiken1975,Bohay2011}
   - Use participants' names (especially online)
   - Pause before speaking
   - Put down your hand if someone has already made your point

> **Manage ``that guy''**
>
> - The moderator's other job{\ldots}
> - Three stickies
> - Interruption bingo
> - https://coast.noaa.gov/ddb/

> **Life online**
>
> - No mixed-mode meetings
> - Do not record the meeting without willing consent
> - Take minutes in a shared document
> - Raise hands digitally
>   - `/hand` in the chat is good
>   - `/hand another budget item` is better

\section*{Appendix: Academic-Industry Partnerships}

An old proverb says, ``If you want to go fast, go alone; if you want to go far,
go together.''  In my experience, this is wrong: going alone is good for a fast
start, but after that, both speed and distance come from having partners.
Researchers and practitioners can each do great things on their own, but both
are better able to solve problems that really matter if they work together.
The ten simple rules listed below may help mitigate the
frustration you encounter as you try to do this.

\subsection*{If you are a researcher in academia{\ldots}}

\subsubsection*{1. Remember that companies work in weeks, not seasons}

Academic semesters are rooted in the seasons of an agricultural era, but
practitioners in industry have to work at a more accelerated pace. In the time
it takes you to write a grant, a company might develop and release two new
versions of their product in order to keep up with their competition. Discuss
timescales with your industrial research partners early on, and be realistic
about how slowly things will proceed.

\subsubsection*{2. Be open}

Research is of no use to practitioners who cannot easily find it and read
it. While Jimmy Wales (the founder of Wikipedia) may not actually have said,
``Open information drives out closed,'' the principle holds: with so much
information freely available on the Internet, any paywall or login barrier put
between you and your hoped-for audience will send a large number of people
elsewhere.

More importantly, these barriers send a clear signal that you do not care if
practitioners read your work or not: as one colleague observed rather sourly,
it's the equivalent of inviting people to your house for dinner and then
expecting them to pay for the drinks.

\subsubsection*{3. Value action over insight}

The goal for practitioners is not to understand the world, but to change it. ``We
know X'' is much less useful to them than ``we can do Y''. When presenting your
findings, you should therefore focus on how someone might act on it.

One way to do this is to add slides titled, ``What Difference Does It Make?'' at
strategic points in your presentations. If you can't think of what to write
next, you may want to rethink what you're focused on.

\subsubsection*{4. Don't hesitate to sacrifice detail for clarity}

Understanding doesn't have to be complete in order to be actionable. For
example, atoms aren't actually little colored balls connected by springs, but
that's still a useful model in organic chemistry. You may need to hedge
conclusions with qualifiers in order to get your work past Reviewer \#3, but
those ``maybes'' and ``howevers'' can often be omitted if they don't change what
practitioners should try next.

\subsubsection*{5. Apologize in advance for the state of academic publishing}

Modern academic publishing isn't actually a conspiracy by a handful of large
companies to line their pockets with government money that could and should be
used to lift researchers out of penury, but it is functionally indistinguishable
from a system that was. The best way to prepare your industry partners for its
Kafkaesque production pipelines and interminable delays is to have them watch
Gilliam's *Brazil*.

\subsection*{If you are a practitioner in industry{\ldots}}

\subsubsection*{6. Remember that universities work in seasons, not weeks}

The timescale mis-match described in Rule \#1 is due in part to the fact that
academic researchers are almost always multi-tasking, and that many of those
tasks are things they've never been trained to do. As students, they juggle
several courses at once (which effectively means that they answer to several
bosses who don't communicate with each other). Later, they are responsible for
teaching, writing grant proposals, and administrative duties.

Collectively, this mean that their ``work week'' is only a few hours long, and
that they will often appear to move at a snail's pace. Be as sympathetic as you
can: they are even less happy with the situation than you are.

\subsubsection*{7. Remember that academic success is measured in publications, not sales}

University presidents routinely make about the economic value of research, but
the only things that truly matter for academic advancement are publication,
publication, and publication. Researchers are not given grants or tenure for
doing things that are ``merely useful'', even if doing so requires a deep
understanding of subtle complexities and months of hard work. For all the jokes
practitioners make about the ivory tower, academic life is hard, uncertain, and
poorly paid. People stay in it for the love of new knowledge; respecting their
priorities is essential to building a productive relationship. (That said,
practical problems often do unlock the door to genuinely new research topics by
pushing researchers out of their comfort zone.)

\subsubsection*{8. Do the background reading}

H.L. Mencken once wrote that, ``There is always a well-known solution to every
human problem—neat, plausible, and wrong.'' Your problem is almost certainly
one of those, and is almost certainly more complex than you first realize. While
Rule \#4 tells researchers to sacrifice detail for clarity, this rule asks
practitioners to make an effort to grasp at least some of that detail so that
you don't waste time reinventing wheels and so that your research partner can
think, work, and talk at full speed.

\subsubsection*{9. Don't overstate what has been learned}

This rule is also a complement to Rule \#4. The ``maybes'' and ``howevers'' that
researchers are so fond of do sometimes matter; if your research partner has
found that regular doses of a new drug seems to slow tumor growth in lab rats,
do not embarrass them by claiming that they have discovered a cure for cancer.

\subsection*{If you are either{\ldots}}

\subsubsection*{10. Apologize in advance for the state of your data}

The final rule applies equally to both researchers and practitioners. Files'
names and locations, the meanings of column headers in tables, how those tables
relate to one another, how missing values are represented and handles:
everything that has made sense to you for years will suddenly seem a little
foolish when you have to explain it to someone else. Apologize in advance, and
then forgive yourself, because no matter how bad your data is, theirs may well
be worse.

\section*{Appendix: Time Management}

You can't produce software (or anything else) without doing some work,
but you can easily do lots of work without producing anything of
value.  Scientific study of overwork goes back to at least the
1890s---see \cite{Robinson2005} for a short, readable summary.  The
most important results for developers are:

1.  Working more than eight hours a day for more than a couple of
    weeks of time lowers your total productivity, not just your hourly
    productivity---i.e., you get less done *in total* (not just per hour)
    when you're in crunch mode than you do when you work regular hours.

1.  Working over 21 hours in a stretch increases the odds of you making a
    catastrophic error just as much as being legally drunk.

These facts have been verified through hundreds of experiments over the course
of more than a century, including some on novice software developers
\cite{Fucci2020}.  The data behind them is as solid as the data linking
smoking to lung cancer.  However, while most smokers will acknowledge that their
habit is killing them, people in the software industry still talk and act as if
science somehow didn't apply to them.  To quote Robinson's article:

\begin{quotation}

  When Henry Ford famously adopted a 40-hour workweek in 1926, he was
  bitterly criticized by members of the National Association of
  Manufacturers.  But his experiments, which he'd been conducting for
  at least 12 years, showed him clearly that cutting the workday from
  ten hours to eight hours---and the workweek from six days to five
  days---increased total worker output and reduced production cost{\ldots}
  the core of his argument was that reduced shift length meant more
  output.

  {\ldots}many studies, conducted by businesses, universities,
  industry associations and the military, {\ldots}support the basic
  notion that, for most people, eight hours a day, five days per week,
  is the best sustainable long-term balance point between output and
  exhaustion.  Throughout the 30s, 40s, and 50s, these studies were
  apparently conducted by the hundreds; and by the 1960s, the benefits
  of the 40-hour week were accepted almost beyond question in
  corporate America.  In 1962, the Chamber of Commerce even published
  a pamphlet extolling the productivity gains of reduced hours.

  But, somehow, Silicon Valley didn't get the memo{\ldots}

\end{quotation}

Everyone knows that designing and building software is a creative act
that requires a clear head, but many people then act as if it was like digging a
ditch.  The big difference is that it's hard to lose ground when digging (though
not impossible).  In software, on the other hand, it's very easy to go backward.
It only takes me a couple of minutes to create a bug that will take hours to
track down later---or days, if someone else is unlucky enough to have to track
it down.  This is summarized in Robinson's first rule:

\begin{quotation}

  Productivity varies over the course of the workday, with the
  greatest productivity occurring in the first four to six hours.
  After enough hours, productivity approaches zero; eventually it
  becomes negative.

\end{quotation}

It's hard to quantify the productivity of programmers, testers, and UI
designers, but five eight-hour days per week has been proven to maximize
long-term total output in every industry that has ever been studied. There is no
reason to believe that ours is any different.

Ah, you say, that's ``long-term total output''.  What about short bursts
now and then, like pulling an all-nighter to meet a deadline?  That's
been studied too, and the results aren't pleasant.  Your ability to
think drops by 25 points for each 24 hours you're awake.  Put it
another way, the average person's IQ is only 75 after one all-nighter,
which puts them in the bottom 5\% of the population.  Two all nighters
in a row and their effective IQ is 50---the level at which people are
usually judged incapable of independent living.

The catch in all of this is that people usually don't notice their
abilities declining.  Just like drunks who think they're still able to
drive, people who are deprived of sleep don't realize that they're not
finishing their sentences (or thoughts).  They certainly don't realize
that they're passing parameters into function calls the wrong way
around or that the reason the tests are failing is that they've
forgotten to redeploy the application again.

``But I have so many assignments to do!'', you (and your teammates) say.
``And they're all due at once!  I *have* to work extra hours to get
them all done!'' Sadly, that is often true: while people in industry
joke that having two bosses means living in hell, most students can
only dream of having just two, since most schools do a lousy job of
coordinating due dates across different courses.  \cite{Tregubov2017}
found strong correlation between the number of projects and the number
of interruptions that developers reported.

Prioritizing is important because most of us are very good at spending
hours on things that don't need to be done and then finding themselves
with too little time for the things that actually count. A little bit
of organizing can help a lot---as \cite{Mark2008} reported,
``{\ldots}people compensate for interruptions by working faster, but
this comes at a price: experiencing more stress, higher frustration,
time pressure and effort.''

\section*{Appendix: Being Fired}

These recommendations are based on the author's experience with DataCamp \cite{Alba2019}
and on many conversations with friends and colleagues.

1. Insist on a record of all conversations.
   The biggest mistake you can make is to assume good faith on the part of those
   who fired you.  In most jurisdictions you have a right to record any phone calls
   you are part of, and if that feels too confrontational, insist on communicating
   by email.  If they insist on communicating by phone or video call, follow up
   immediately with an email summary and make sure you send a copy to your personal
   account.

2. Pause before speaking, posting, or tweeting.
   If possible, have someone you trust look everything over before you say it or
   send it.  (Don't use someone who still works for the company, even if they are
   your closest friend: it puts them in a legally and morally difficult position.)

3. Keep your public statements brief.
   People may care, but most won't care as much as you do.  A simple recitation of
   facts is usually damning enough.

4. If you want to correct something online, add a timestamped amendment.
   Do not just take down or edit anything you have written.  If you do, you will be
   accused of rewriting history, and muddied waters only help whoever fired you.
   (Also, be prepared for them to dig through everything you've ever said online
   and re-post parts selectively to discredit you.)

5. Speak directly to all the issues rather than omitting or ignoring things you'd rather not discuss.
   Your honesty is your greatest asset, and it's hypocritical to criticize your
   opponents for spin or selective reporting if you're doing it too.

6. Don't sign any agreement that might prevent you from speaking about moral or legal concerns.
   And make sure the agreement explicitly excludes your concerns before signing it.
   (Yes, it's very privileged of me to be able to say this: someone whose
   immigration status, essential health benefits, or family income is being
   threatened may not have a choice.  That is why I think people who do have a
   choice also have an obligation to fight.)

7. Don't cite the law unless your lawyer tells you to.
   It probably doesn't mean what you think it means, and they almost certainly do
   have lawyers on their side who will seize on any mis-statement or mistake you
   make.

8. Don't try to get them to acknowledge that they were wrong.
   This probably wouldn't have happened if they were the sort of people who could.

9. Go for long walks, cook some healthy meals, pick up the guitar you haven't touched in year.
   Do anything that requires you to focus on something else for a while.  This
   isn't just for your mental health: exhausted people make poor decisions, and you
   need to be at the top of your game.

10. Remember that it's OK to cry.
    Because we all do.

\section*{Appendix: Moving On}

Sooner or later it will be time to hand over your project or wind it down.
I hope this advice will help.

1.  Be sure you mean it.
    Letting go will be hard on you,
    but *not* letting go will be even harder on your successors,
    so be sure you're actually going to do it.

2.  Do it when others think it's time.
    You are the last person to realize when you're too tired to be coding.
    You will often be the last person to realize that you ought to be moving on,
    so ask people and pay attention to what they say.

3.  Tell people what, when, and why.
    Tell people that you're leaving and what the succession plan is as soon as possible
    (which in practice means ``as soon as you think you won't have to revise what you have said publicly'').

4.  Don't pick a successor by yourself.
    You may have strong opinions about who should succeed you,
    but you should still check those opinions with someone more objective.

5.  Train them before you go.
    Share tasks with your successor for a few days or weeks:
    they will get to see how things actually work,
    and you'll discover things you would otherwise forget to tell them.
    Go on holiday for a week and leave your successor temporarily in charge.
    You'll discover even more things you would otherwise forget to pass on.

6.  When you leave, leave.
    It may be tempting to continue to have a role in the organization,
    but that usually leads to confusion,
    since people are used to looking to you for answers.

7.  Have some fun before you go.
    Don't just slog your way to the end and turn out the lights:
    take that back burner project you've been dreaming about and give it a shot.

8.  Reflect on what you learned.
    Whatever you have left will almost certainly not be the last thing you ever do.
    Take some time to think about what you could have done differently,
    write it down,
    and then move on:
    obsessing over only-ifs and might-have-beens won't help anyone.

9.  Remember the good things too.
    Many people are uncomfortable being praised---so uncomfortable that
    English doesn't actually have an antonym for ``mistake''.
    Thank the people who made what you did successful,
    and give yourself a little credit as well,
    even if you're Canadian.

10. Do something next
    People who start things usually aren't good with idleness,
    and idleness tends not to be good for them,
    so when you leave,
    leave for something,
    even if it's something small.

\bibliography{codebender}

\end{document}
