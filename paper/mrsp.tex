\documentclass[10pt,letterpaper]{article}
\usepackage[top=0.85in,left=2.75in,footskip=0.75in]{geometry}

% amsmath and amssymb packages, useful for mathematical formulas and symbols
\usepackage{amsmath,amssymb}

% Use adjustwidth environment to exceed column width (see example table in text)
\usepackage{changepage}

% cite package, to clean up citations in the main text. Do not remove.
\usepackage{cite}

% Use nameref to cite supporting information files (see Supporting Information section for more info)
\usepackage{nameref,hyperref}

% line numbers
\usepackage[right]{lineno}

% ligatures disabled
\usepackage{microtype}
\DisableLigatures[f]{encoding = *, family = * }

% color can be used to apply background shading to table cells only
\usepackage[table]{xcolor}

% array package and thick rules for tables
\usepackage{array}

% Use listings instead of verbatim
\usepackage{listings}
\lstset{
  basicstyle=\ttfamily,
  escapeinside=!!
}

% enumerate package lets us use letters instead of numbers
\usepackage{enumerate}

% create "+" rule type for thick vertical lines
\newcolumntype{+}{!{\vrule width 2pt}}

% create \thickcline for thick horizontal lines of variable length
\newlength\savedwidth
\newcommand\thickcline[1]{%
  \noalign{\global\savedwidth\arrayrulewidth\global\arrayrulewidth 2pt}%
  \cline{#1}%
  \noalign{\vskip\arrayrulewidth}%
  \noalign{\global\arrayrulewidth\savedwidth}%
}

% \thickhline command for thick horizontal lines that span the table
\newcommand\thickhline{\noalign{\global\savedwidth\arrayrulewidth\global\arrayrulewidth 2pt}%
\hline
\noalign{\global\arrayrulewidth\savedwidth}}

\usepackage{color}

% Remove comment for double spacing
%\usepackage{setspace}
%\doublespacing

% Text layout
\raggedright
\setlength{\parindent}{0.5cm}
\textwidth 5.25in
\textheight 8.75in

% Bold the 'Figure #' in the caption and separate it from the title/caption with a period
% Captions will be left justified
\usepackage[aboveskip=1pt,labelfont=bf,labelsep=period,justification=centering,singlelinecheck=off]{caption}
\renewcommand{\figurename}{Fig}

% Use the PLoS provided BiBTeX style
\bibliographystyle{plos2015}

% Remove brackets from numbering in List of References
\makeatletter
\renewcommand{\@biblabel}[1]{\quad#1.}
\makeatother

% Leave date blank
\date{}

% Header and Footer with logo
\usepackage{lastpage,fancyhdr,graphicx}
\usepackage{epstopdf}
\pagestyle{myheadings}
\pagestyle{fancy}
\fancyhf{}
\setlength{\headheight}{27.023pt}
\rfoot{\thepage/\pageref{LastPage}}
\renewcommand{\footrule}{\hrule height 2pt \vspace{2mm}}
\fancyheadoffset[L]{2.25in}
\fancyfootoffset[L]{2.25in}
\lfoot{\sf PLOS}

% Allow Markdown while prototyping
\usepackage[hashEnumerators,smartEllipses]{markdown}


\begin{document}
\vspace*{0.2in}

\begin{flushleft}
{\Large
\textbf\newline{Good Enough Practices for Managing Research Software Projects}
}
\newline
\\
{Greg Wilson}\textsuperscript{1{\ddag}}
\\
\bigskip
\textbf{1} Third Bit\\
{\ddag} Corresponding author, gvwilson@third-bit.com.
\end{flushleft}

\section*{Introduction}

\begin{markdown}

- \cite{Wilson2014,Wilson2017} focused on technical aspects of building research software
  - Version control, file organization, automating repeated tasks
- But also touched on some management aspects
  - Create a shared to-do list, deciding on communication channels, creating checklists
- This article introduces practices for managing a team of a dozen people working together to build research software
- You don't have to invent this yourself \url{https://www.askamanager.org/}

\end{markdown}

\subsection*{Acknowledgments}

FIXME: Daniel Standage

\section{Governance}

\begin{markdown}

- Every group has a power structure
  - Only question is whether it is explicit and accountable, or implicit and unaccountable \cite{Freeman1972}
- <https://communityrule.info/> describes some options and \cite{Fogel2021} describes some more
- GitHub's [Minimum Viable Governance](https://github.com/github/MVG) guidelines ducks the problem

\begin{quotation}
  \textbf{2.1. Consensus-Based Decision Making}.
  Projects make decisions through consensus of the Maintainers.
  While explicit agreement of all Maintainers is preferred, it is not required for consensus.
  Rather, the Maintainers will determine consensus based on their good faith consideration of a number of factors,
  including the dominant view of the Contributors and nature of support and objections.
  The Maintainers will document evidence of consensus in accordance with these requirements.
\end{quotation}

- In practice, "consensus-based" usually means "governance by the self-confident and stubborn"
  - Excludes a lot of people
- Brooks advocated a "chief surgeon" model in the 1970s \cite{Brooks1995}
  - Often referred to in open source as "benevolent dictator"
  - But as a Latin American colleague pointed out, there's no such thing{\ldots}
- Disparaged since then
  - But 80\% of projects on GitHub are "hero projects" \cite{Majumder2019}
  - 5\% or less of people responsible for 95\% or more of interactions
  - "Heroes" commit far fewer bugs than other contributors
- Common and sensible in young or small projects
  - Especially where specialized domain knowledge is required
  - Or where one person (the PI) actually does have all the authority
  - But brittle (founder can move on)
  - Leads to emergence of unofficial (i.e., unaccountable) sub-leaders as project grows
  - As project grows to include other contributors, they'll want a say
- Elected representation at the very large end
  - With explicit rules for suffrage
- In between, best to use Martha's Rules \cite{Minahan1986}
- Anyone can put forward a proposal by filing it at least 24 hours before a scheduled meeting
  - One-line summary
  - Background information
  - Concrete proposal
  - Pros and cons
  - Alternatives
- At most two pages
  - Preferably shorter
- A quorum is established if half or more of voting members are present (see below)
- The meeting may not discuss or vote on a proposal unless its sponsor (or their delegate) is present.
- The sponsor presents the proposal
  - Usually unnecessary since everyone should have read it
- Members cast a *sense vote*: support, neutral, or oppose
  - If everyone supports or is neutral, go immediately to a binary vote with no further discussion
  - If a majority is opposed or neutral, send proposal back to sponsor for further work
- If a minority of members oppose, set a timer for 10 minutes of moderated discussion
- Then call a final binary vote in which everyone must support or oppose
  - No neutral votes allowed, since either we're going to do this or not
- If a majority support, the proposal is accepted
  - Otherwise, it is returned to the sponsor for further work

\begin{quotation}
- Needs to be an explicit rule about how to become a member
  - And a rule about when and how someone stops being a member (FIXME: forward ref to HR)
- This may align with who gets authorship credit on software publications
\end{quotation}

FIXME: example proposal

\end{markdown}

\section{Reporting}

\begin{markdown}

- Progress, plans, and problems go up
- Context, responsibilities, and priorities come down

\end{markdown}

\section*{Conclusion}

FIXME

\nocite{*}
\bibliography{codebender}

\end{document}
